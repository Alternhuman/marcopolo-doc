% !TEX encoding = UTF-8 Unicode
%%%%%%%%%%%%%%%%%%%%%%%%%%%%%%%%%%%%%%%%%
% Requisitos del protocolo de descubrimiento de servicios a crear
% 
% Autoría original:
% Linux and Unix Users Group at Virginia Tech Wiki 
% (https://vtluug.org/wiki/Example_LaTeX_chem_lab_report)
% Proporcionada por http://www.LaTeXTemplates.com
%
% Licencia:
% CC BY-NC-SA 3.0 (http://creativecommons.org/licenses/by-nc-sa/3.0/)
%
%%%%%%%%%%%%%%%%%%%%%%%%%%%%%%%%%%%%%%%%%

%----------------------------------------------------------------------------------------
%	PACKAGES AND DOCUMENT CONFIGURATIONS
%----------------------------------------------------------------------------------------

\documentclass{article}
\newcommand{\nombre}{MarcoPolo}
\newcommand{\titulo}{\nombre{}: Protocolo de descubrimiento de servicios}
\newcommand{\autor}{Diego Martín Arroyo}



\usepackage{graphicx} % Requerido para la inclusión de imágenes
\usepackage{listings} % Requerido para la inserción de código
\usepackage[numbers]{natbib} % Requerido para cambiar el estilo de la bibliografía a APA. [numbers] para eliminar problemas de compatibilidad. Ver http://www.multiasking.com/blog/package-natbib-error-bibliography-not-compatible-with-author-year-citation/


\usepackage[T1]{fontenc} % Codificación de las fuentes utilizadas
%\usepackage[utf8]{inputenc} % Codificación de caracterers de entrada en UTF-8 (necesario para los caraceteres propios del español). Comentado dado que es ignorado con el compilador de XeLaTeX
\usepackage[spanish]{babel} % Español como idioma principal del texto (permite hyphenation de palabras al final de una línea)

\usepackage{courier} % Requerido para el uso de la fuente Courier en los listings y demás código incluido
\usepackage[usenames,dvipsnames]{color} % Requerido para el uso de colores

%Metadatos del PDF e inclusión de referencias "clicables"
\usepackage[pdfauthor={\autor},
            pdftitle={\titulo},
            pdfproducer={XeLaTeX with hyperref},
            pdfcreator={XeLaTeX}]{hyperref}

\usepackage{url}
\usepackage{float} % Requerido para el posicionamiento absoluto de las imágenes. Ver http://en.wikibooks.org/wiki/LaTeX/Floats,_Figures_and_Captions
\usepackage[raggedright]{titlesec} % Evita la unión con guión de los \(sub)*section. Ver http://tex.stackexchange.com/questions/24777/how-to-disable-hyphenation-in-all-section-and-subsection-titles
%And if You used titleformat from titlesec to style titles -- You can do it explicitly: \titleformat{\section}{\color{teal}\LARGE\raggedright}{\hspace{0.5cm}\thetitle\‌​hspace{0.5cm}}{0cm}{} (note the \raggedright)

%\usepackage{titlesec}
%\newcommand{\sectionbreak}{\clearpage} % See http://tex.stackexchange.com/questions/9497/start-new-page-with-each-section

% Márgenes
\topmargin=-0.45in
\evensidemargin=0in
\oddsidemargin=0in
\textwidth=6.5in
\textheight=8.5in
\headsep=0.25in

\graphicspath{{./img/}}
\selectlanguage{spanish}

\newcommand*\lstinputpath[1]{\lstset{inputpath=#1}}

%\setlength\parindent{0pt} % Removes all indentation from paragraphs

%\renewcommand{\labelenumi}{\alph{enumi}.} % Descomentar para que la enumeración en el entorno enumerate sea mediante letras en vez de números




%Lenguajes de los que mostrar código
\lstloadlanguages{Java,XML,Python,JavaScript}

%Ajustes para Java
\lstset{
    language=java,
    frame=single, % Un marco simple alrededor del código
    basicstyle=\small\ttfamily, % Utilizar fuente true type pequeña
    keywordstyle=[1]\color{Blue}\bf, % Funciones en negrita y azul
    keywordstyle=[2]\color{Purple}, % Argumentos en morado
    keywordstyle=[3]\color{Blue}\underbar, % Funciones personalizadas subrayadas en azul
    identifierstyle=, % Nada especial acerca de identificadores
    commentstyle=\usefont{T1}{pcr}{m}{sl}\color{Green}\small, % Los comentarios se renderizan en fuente pequeña verde
    stringstyle=\color{Purple}, % Cadenas en morado
    showstringspaces=false, % No se muestran los espacios entre cadenas
    tabsize=5, % 5 espacios por tabulado
    %
    % Put standard Perl functions not included in the default language here
    %morekeywords={rand},
    %
    % Put Perl function parameters here
    %morekeywords=[2]{on, off, interp},
    %
    % Put user defined functions here
    %morekeywords=[3]{test},
   	%
    morecomment=[l][\color{Blue}]{...}, % Line continuation (...) like blue comment
    numbers=left, % Número de línea a la izquierda
    firstnumber=1, % Número de línea comienza en 1
    numberstyle=\tiny\color{Blue}, % Los números de línea son azules y pequeños
    stepnumber=5, % Los números de línea van de 5 en 5
    breaklines=true % Salto de línea si el texto no entra. See http://stackoverflow.com/a/1875803
}

%\usepackage{times} % Uncomment to use the Times New Roman font

%----------------------------------------------------------------------------------------
%	DOCUMENT INFORMATION
%----------------------------------------------------------------------------------------

\title{\titulo}
\author{\autor}
\date{\today} % Fecha

%\lstinputpath{../../src/}

\newcommand{\javacode}[4]{
	\lstinputlisting[caption=#2,label=#1, firstline=#3, lastline=#4]{#1.java}
}

\newcommand{\txtcode}[4]{
	\lstinputlisting[caption=#2, label=#1, firstline=#3, lastline=#4]{#1.txt}
}

\newcommand{\pythoncode}[4]{
	\lstinputlisting[caption=#2, label=#1, firstline=#3, lastline=#4]{#1.py}
}

\begin{document}

\maketitle

\begin{abstract}
El protocolo \nombre{} permite el descubrimiento de máquinas ofreciendo una serie de servicios consumibles por el sistema distribuido, permitiendo publicar y solicitar nodos, obtener información del estado de nodos y una serie de funcionalidades de diagnóstico.
\end{abstract}

\section{Motivación}

El protocolo surge de la necesidad de minimizar la configuración manual que debe realizar un administrador en cada nodo y las ventajas que ofrece la integración automática de un equipo en la red de nodos

\section{Descripción de la interacción entre partes}

\begin{itemize}
  \item\texttt{Discover} (Marco) Descubrimiento de todos los nodos en un grupo multicast concreto
  \item\texttt{Node-Alive} (Polo) Envío de toda la información sobre un nodo en respuesta a un mensaje \texttt{Discover}
  \item\texttt{Request-for: <información>} Servicios, coordinador, etc
  \item\texttt{Offer: <Servicio>} Un nodo puede indicar los servicios que ofrece
  \item\texttt{OK-For: <Servicio>} Comando diseñado para aquellos servicios que necesiten conocer los nodos interesados en los mismos (opcional) 
  \item\texttt{Request-Service}
  \item\texttt{Need-Copy-Of} Los nodos pueden conseguir copias de los paquetes \texttt{.mp} que posea otro nodo, evitando la necesidad de copiar uno a otro 
  \item\texttt{Bye} Un nodo se desconecta de la red y notifica este cambio
\end{itemize}

Todo el tráfico se realiza por \textit{multicast}, permitiendo que varios sistemas trabajen sobre la misma red de forma solapada.

\section{Seguridad}

\nombre trabaja con certificados que permiten verificar la confianza de un nodo dado. Mediante la directiva \texttt{Accept-Cert-Only} es posible aceptar únicamente nodos fiables

\section{Integración}

El protocolo es integrable en código mediante \textit{bindings} que se comunican con la instancia de \textbf{\nombre} en la máquina.
Existen \textit{bindings} con Python, C/C++ y Java.

\section{Utilidades}

Se incluyen diferentes utilidades para trabajar con \nombre desde una consola.

\begin{itemize}
  \item marcodiscover [-s service] para descubrir nodos
  \item marcodig <ip> para encontrar información sobre un nodo
  \item poloregister para añadir un nuevo servicio a la oferta sin tener que crear un fichero de configuración
\end{itemize}







%%	BIBLIOGRAPHY
%%----------------------------------------------------------------------------------------
%

\bibliographystyle{ieeetr}
%
\bibliography{marcopolo}
%%----------------------------------------------------------------------------------------
%

\end{document}
